\section{Forward and Inverse Kinematic}

\subsection{Forward Kinematics}

Forward kinematics is about finding the position and orientation of a tool point, which is generally the robot end-effector. The calculation is based on specified joint angles as parameters. 

\subsection{Inverse Kinematics}

The inverse kinematics is the opposite of the forward kinematics. This means inverse kinematics refers to that joint angles are calculated, in order for the tool point at the end-effector to be in the desired position and orientation. This way of behaviour is more relevant to most real world applications and tasks performed by robots.

One of the big differences between forward and inverse kinematics is that forward kinematics usually has one distinct solution. Setting all the angles of the joints for a robot arm to certain values will generally result that the tool point will end in one particular position. However, inverse kinematics is quite different. Depending on the situation and design structure of the robot arm, there are will be many ways to position the robot arm joints and angle them in

