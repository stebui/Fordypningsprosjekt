\section{Forward and Inverse Kinematic}

\subsection{Forward Kinematics}

Forward kinematics is about finding the position and orientation of the tool point, which is generally the robot end-effector, from the joint angles along the robot arm. The calculation is based on specified joint angles as parameters \cite{forward_kinematics}. The position is found by going forward from the base of the robot toward the end-effector. 

\subsection{Inverse Kinematics}

The inverse kinematics is the opposite of the forward kinematics. This means inverse kinematics refers to what particular angles should be set for all the robot joints, in order for the end-effector to be in the desired position and orientation. This way of behaviour is more relevant to most real world applications and tasks performed by robots \cite{inverse_kinematics}. 

One of the big differences between forward and inverse kinematics is that forward kinematics usually has one distinct solution. Setting all the angles of the joints for a robot arm to certain values, will generally result that the tool point will end in one particular position. However, inverse kinematics is quite different. Depending on the situation and design structure of the robot arm, there could be many ways to position the joints of a robot arm and angle them to reach a specific point in workspace. A good example of this would be to use your own arm to reach a point in space, from as many different possible ways you can position your shoulder, wrist or elbow joint. Inverse kinematics can yield multiple solutions, but that does not necessarily mean all solution is feasible. Each solution need to be validate against criteria given by the robot specifications and workspace. These solutions have to satisfy the constraints of each joint and valid movements in a reachable workspace.
