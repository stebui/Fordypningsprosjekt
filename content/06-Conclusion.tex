\chapter{Conclusion}

\section{Result}
An interactive GUI for the control system has been implemented. The method uses different modules to achieve a working control system for vehicles and tools, represented as geometrical models. It has been made to adapt to models that are dynamically linked to the library in GeoMod. This has been done under consideration for an autonomous mode employed by other systems. 

The designed GUI enables control over position and orientation of a model in a three-dimensional space. It is arranged such that translation and rotation can be done along the three Cartesian axes and with the possibility for spherical motion in future development. Options to move relative to a chosen reference system has been implemented. This allows models and tools to follow the corresponding reference, existing or added, in GeoMod. Value information in the form of their respective coordinate system is displayed as useful instruments. The transition of movement is no longer predetermined as it is now possible to configure displacement to all directions. Controlling model represented as a vehicle with a tool, enables control choices for forward and inverse kinematics. The implemented control system supports interaction between models and tracking path with or without a tool.

\section{Discussion}

Much of the functionality had to be revised to create a working control system. A major challenge was to solve how the data needed was passed between different modules, as the implementations originally were done independently. As more module and functionality is added, the increase of connection and interactions gets more complicated to handle for a working control system. Finally, a GUI that emphasises the structure, intuitiveness and a compact design had to be kept in mind for the user to utilise. The GUI is arranged such that new features can be implemented, and also further develop awaited functions needed. Some of these functionality has been mentioned throughout this project, e.g. spherical translation, joystick input and appropriate interface for cameras. The control system for tools was to some degree implemented in the control systems. To base this system simply on one single model with corresponding tool, limits the feasibility of testing and assumption for developing a functional control system. A generalised GUI primarily for tools with the use of forward and inverse kinematics was not developed under this project. However, the framework and the approach has been reviewed during this project to serve as a foundation for further implementation in the future.

The successfulness of this assignment can be assessed on the basis of how the proceeded and achieved efforts have resolved or contributed to the objectives in \cref{chap:objectives}. With the newly implemented control system, the project has made some progress towards a unified system that is suited to test and demonstrate autonomous interactions of vehicles and tools.