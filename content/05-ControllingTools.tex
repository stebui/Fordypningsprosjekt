\chapter{Controlling Vehicle Tools}

So far, a control system for vehicles has been implemented with a GUI. This chapter will have a look on how to proceed with embedding the control system of the associated tools from a vehicle. Since only one model contains an end-effector, the generalised system will be based on the serial manipulator model (\Cref{fig:SerialManipulator}).

\begin{figure}[ht]
    \centering
    \includegraphics[height=8cm]{images/GeometricModel.png}
    \caption[Geometric model of a serial manipulator]{Geometric model of a serial manipulator.}
    \label{fig:SerialManipulator}
\end{figure}

\section{Kinematics}

It was mentioned that from the work of Martin Lygre Fuglevik, the control system for desired position and orientation of objects along a network of paths was partially implemented. Most of the implementation worked as it should, but some instances regarding rotation in motion to a particular path did not work as prompted. Some small alteration had to be done in the model in order to continue the implementation of the control system. 

\begin{figure}[ht]
    \centering
    \includegraphics[height=7cm]{images/Diagram_Serial_Manipulator.png}
    \caption[Kinematic connection of each part of the serial manipulator model]{Kinematic connection of each part of the serial manipulator model.}
    \label{fig:KinematicsSerialManipulator}
\end{figure}


The \Cref{fig:KinematicsSerialManipulator} illustrates how the communication between each joint is conducted by forward and inverse kinematics on the implemented model. This is done by having variables over the position, rotation and angle on each part of the model. By applying forward kinematics, the joints angle is independently controlled for each parts beginning from the position of the linked mechanism, the base, to the end-effector, i.e. gripper. The function \textit{dirKin()} is an existing implemented function that calculates joint angles of all the joints, when one of the parts is moved. This makes a lot of unnecessary computation considering joints that are not affected by the movements. However, this was neglected as it did not prevent the fulfilment of the current objectives for the project. This should be taken into account in the future for optimisation. 

Forward kinematics works well for a desired end position for the end-effector, but is rather impractical to control when following a path. With inverse kinematics, the position of the base and joint angle is determined from the position of the gripper. Working with inverse kinematics makes it possible to have a desired position for the gripper while the base is under displacement as long as it is within the workspace. Note, it also applies to a displacement of a path. This method of control is very helpful for an AUV, considering currents and other significant disturbances in the ocean. In addition, moving a tool along a path can be achieved, as the joint angles are calculated and set during the series of translation and rotation of the gripper. A control interface for a tool with forward and inverse kinematics can be applied to any model given with a base and an end-effector. 

\begin{figure}[ht]
  \centering
  \subfloat[Forward kinematics]{\includegraphics[width=0.31\textwidth]{images/forward.png}\label{fig:forward}}
  \hfill
  \subfloat[Inverse kinematics]{\includegraphics[width=0.31\textwidth]{images/inverse.png}\label{fig:inverse}}
  \hfill
  \subfloat[Motion control]{\includegraphics[width=0.31\textwidth]{images/Motion.png}\label{fig:motion}}
  \caption[Control panels implemented by Martin Lygre Furevik]{Control panels implemented by Martin Lygre Furevik.}
  \label{fig:martincontrolpanels}
\end{figure}

The implementation of the inverse kinematics solver for this specific model was developed by Professor Sven Fjeldaas \cite{sven}. In Martin Lygre Fureviks' master thesis, the inverse kinematics solver was implemented in the model with a corresponding control panel for the inverse kinematics. A similar control panel for forward kinematics was also made for the model. Furthermore, a motion controller for the robot was created to move the gripper along a path. The controls for the forward, inverse kinematics and motion can be seen in \Cref{fig:martincontrolpanels}. The implementation of the motion controller is flexible and generalised, requiring only a tool and a path to work. This is done by connecting a pointer to the tool variable from an end-effector model, like the gripper, and a pointer to the path variables from a model. Any given model can, therefore, be freely moved along a given path. For tools, the motion on the path calls the inverse kinematics implemented in the model to regulate the relation between the vehicle and the tool. Combining the motion controller with the implemented control system for this project, tracking a path while the base or the path is in motion is now possible to demonstrate in GeoMod (\Cref{fig:displacement}). 


\begin{figure}[ht]
  \centering
  \subfloat[The gripper is following a path.]{\includegraphics[width=0.4\textwidth]{images/serial_1.png}\label{fig:s1}}
  \hfill
  \subfloat[The gripper is still following the path, during displacement.]{\includegraphics[width=0.4\textwidth]{images/serial_2.png}\label{fig:s2}}
  \caption[Displacement of the base of the serial manipulator model]{Displacement of the base of the serial manipulator model.}
  \label{fig:displacement}
\end{figure}