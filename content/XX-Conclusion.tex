\chapter{Conclusion}

\section{Result}
An interactive graphical user interface for the control system has been implemented. The method uses different modules to achieve a working control system for vehicles and tools, represented as geometrical models. It has been made to adapt to models that are dynamically linked to the library in GeoMod. This has been done under consideration for an autonomous state employed by other systems. 

The designed GUI enables control over position and orientation of a model in a three-dimensional space. It is arranged so that translation and rotation can be done along the three Cartesian axes and with the possible for spherical motion in the future. Options to move relative to a chosen reference has been implemented. This allows models to follow a given reference system, existing or added, in GeeMod. Value information in the form of their respective coordinate system is displayed as useful instruments. The transition of movement is no longer predetermined as it is now possible to select how much displacement to either direction. Controlling model represented as a vehicle with a tool, enable control choices for forward or inverse kinematics. The implemented control system supports interaction between models and tracking path with or without a tool.

\section{Conclusion}

A lot of the functionality had to be revised to create a working control system. A big challenge was how the data needed to be passed between different modules, as it was implemented to work independently.  As more module and functionality is added, the increase of connection and interactions gets more complicated to handle for working control system.  Finally, a GUI that emphasised on structure, intuitive and compact design had to be kept in mind for the user to utilise. The GUI is arranged so that many new features can be implemented and further develop awaited functions needed. Some of those functionality has been mentioned throughout this project, especially spherical translation and appropriate interface for cameras. Since all projects have a deadline is it difficult to say how much can be done during a project. The control system for tools was somewhat implemented in the control systems. To simply have one model that has a tool, limits the feasibility of testing and assumption for developing a functional control system. A generalise GUI for tools with the use of forward and inverse kinematics was not developed under this project. But the framework and the approach has been reviewed during this project to proceed the implementation. 

This project can be concluded from how the proceeded and achieved intentions have resolved the objectives, given in [link], in this assignment. Based on the objectives, one could somewhat say all parts was completed with a degree of some minor implementation left out.
